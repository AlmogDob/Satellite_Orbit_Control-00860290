\documentclass[11pt, a4paper]{article}
\usepackage{amsmath, amssymb, titling}
\usepackage[margin=2cm]{geometry}

\renewcommand\maketitlehooka{\null\mbox{}\vfill}
\renewcommand\maketitlehookd{\vfill\null}
% \usepackage{enumitem}
\usepackage{graphicx}
\usepackage{float}

% \usepackage{minted}
% \usepackage{xcolor}

\title{Satellite Orbit Control \\ HW2}
\author{Almog Dobrescu\\\\ID 214254252}

\begin{document}

\maketitle

\thispagestyle{empty}
\newpage
\setcounter{page}{1}

\tableofcontents
\vfil
\listoffigures
\newpage

\section{Given}
\begin{equation*}
    \begin{matrix}
        T_1=100\left[min\right] = 6\cdot10^3\left[sec\right] && T_2 = T_1 = 6\cdot10^3\left[sec\right] \\
        e_1 = 0 && e_2 = 0 \\
        a_1 = \sqrt[3]{\frac{\mu T_1^2}{4\pi^2}} = 7.1366\cdot10^3\left[km\right] && a_2 = a_1 = 7.1366\cdot10^3\left[km\right]
    \end{matrix}\\
\end{equation*}
\begin{equation*}
    \alpha=\Delta i = 0.01^\circ
\end{equation*}
In CW fram with origin at Satellite \#1 and at $t=0$:
\begin{equation*}
    \begin{matrix}
    \begin{pmatrix}
        x_2(0)=0 \\ y_2(0)=-1 \\ z_2(0)=1
    \end{pmatrix}\left[km\right] &&
    \begin{pmatrix}
        \dot{x}_2(0)=?? \\ \dot{y}_2(0) = ?? \\ \dot{z}_2(0)<0
    \end{pmatrix}
    \end{matrix}
\end{equation*}
Desired:
\begin{equation*}
    \begin{matrix}
    \begin{pmatrix}
        x_2(t_1)=0 \\ y_2(t_1)=0 \\ z_2(t_1)=0
    \end{pmatrix} &&
    \begin{pmatrix}
        \dot{x}_2(t_1)=0 \\ \dot{y}_2(t_1)=0 \\ \dot{z}_2(t_1)=0
    \end{pmatrix}
    \end{matrix}
\end{equation*}
Limitations:
\begin{equation*}
    \Delta v = t \begin{pmatrix}
        0\\
        \Delta v_y\\
        \Delta v_z
    \end{pmatrix}
\end{equation*}

\section{A}
General equations of motion in CW fram:
\begin{equation}
    \left\{\begin{array}{l}
        \ddot{x}-2n\dot{y}-3n^2x=f_x\\
        \ddot{y}+2n\dot{x}=f_y\\
        \ddot{z}+n^2z=f_z
    \end{array}\right.
\end{equation}
The velocities:
\begin{equation}
    \left\{\begin{array}{l}
        u_x = \dot{x}-ny(t)\\
        u_y = \dot{y}+nx(t)\\
        u_z = \dot{z}
    \end{array}\right.
\end{equation}
The solution without external forces (i.e. $\vec{f}=\vec{0}$):
\begin{equation}
    \left\{\begin{array}{l}
        \displaystyle x(t) = \left(4-3\cos(nt)\right)\cdot x_0 + \frac{\dot{x}_0}{n}\sin(nt)+\frac{2}{n}\left(1-\cos(nt)\right)\cdot\dot{y}_0 \\
        \displaystyle y(t) = 6\left(\sin(nt)-nt\right)\cdot x_0 + y_0 + \frac{2}{n}\left(\cos(nt)-1\right)\cdot\dot{x}_0 + \left(4\sin(nt)-3nt\right)\frac{\dot{y}}{n} \\
        \displaystyle z(t)=z_0\cos(nt)+\frac{\dot{z}_0}{n}\sin(nt)
    \end{array}\right.
\end{equation}
Because the two satellite have the same period then the motion is periodical. The condition for periodical motion is:
\begin{equation}
    \dot{y}_0=-2nx_0
\end{equation}
So the equations of motion becomes:
\begin{equation}
    \left\{\begin{array}{l}
        \displaystyle x(t) = x_0\cos(nt) + \frac{\dot{x}_0}{n}\sin(nt) \\
        \displaystyle y(t) = y_0 + \frac{2}{n}\left(\cos(nt)-1\right)\cdot\dot{x}_0 -2x_0\sin(nt) \\
        \displaystyle z(t)=z_0\cos(nt)+\frac{\dot{z}_0}{n}\sin(nt)
    \end{array}\right.
\end{equation}
% The required maneuver time is the time required for the position in the \emph{z} direction to become zero:
% \begin{align}
%     z_0\cos(nt)+\frac{\dot{z}_0}{n}\sin(nt) &= 0 \\
%     \cos(nt)+\frac{\dot{z}_0}{n}\sin(nt) &= 0 \\
%     \sqrt{1^2+\frac{\dot{z}}{n}}\cos\left(nt + \arctan\left(\frac{\displaystyle -\frac{\dot{z}_0}{n}}{1}\right)\right) = 0 \\
%     nt +\arctan\left(\frac{\displaystyle -\frac{\dot{z}_0}{n}}{1}\right) = \frac{\pi}{2} + \pi k
% \end{align}
\begin{equation}
    \left\{\begin{array}{l}
        \displaystyle\dot{x}(t) = -nx_0\sin(nt) + \dot{x}_0\cos(nt) \\
        \displaystyle\dot{y}(t) = -\frac{2\dot{x}_0}{n}\left(n\sin(nt)+1\right)-2nx_0\cos(nt) \\
        \dot{z}(t) = -nz_0\sin(nt) + \dot{z}_0\cos(nt)
    \end{array}\right.
\end{equation}
From the angle between the plains we can claculate $\dot{z}_2(0)$:
\begin{align}
    \tan\alpha &= \frac{z_{max}}{a_1} = \frac{\displaystyle\sqrt{z_2(0)^2+\left(\frac{\dot{z}_2(0)}{n}\right)^2}}{a_1} \\
    \frac{\dot{z}_2(0)}{n} &= \underbrace{-}_{\displaystyle\dot{z}_2(0)<0}\sqrt{\left(a_1\cdot\tan(\alpha)\right)^2-z_2(0)^2} \\
    \frac{\dot{z}_2(0)}{n} &= -0.7426;
\end{align}
The CW equations in state-space form:
\begin{equation}
    \begin{matrix}
        \text{x-y plane:} && \text{z-direction:} \\
        \begin{matrix}
            x1 = x \\
            x2 = \dot{x}
        \end{matrix} && z1 = z \\
        \begin{matrix}
            y1 = y \\
            y2 = \dot{y}
        \end{matrix} && z2 = \dot{z}
    \end{matrix}
\end{equation}
The homogeneous solution with no drift condition:
\begin{equation}
    \vec{x}(t) = \Phi_{(t,t_0)}\vec{x}_0
\end{equation}
\begin{equation}
    \begin{array}{l}
        \text{x-y plane:} \\
        % \Phi_{(t,t_0)} = \left(\begin{array}{cc|cc}
        %     4-3\cos(n\tau) & 0 &\displaystyle \frac{1}{n}\sin(n\tau) &\displaystyle \frac{2}{n}\left(1-\cos(n\tau)\right) \\ 
        %     6\left(\sin(n\tau)-n\tau\right) & 1 &\displaystyle \frac{2}{n}\left(\cos(n\tau)-1\right) &\displaystyle \frac{1}{n}\left(4\sin(n\tau-3n\tau)\right)\\ &&& \\ \hline &&& \\
        %     3n\cdot\sin(n\tau) & 0 & \cos(n\tau) & 2\sin(n\tau) \\
        %     6n\left(\cos(n\tau)-1\right) & 0 & -2\sin(n\tau) & 4\cos(n\tau)-3 
        \Phi_{(t,t_0)} = \left(\begin{array}{cc|cc}
            \cos(n\tau) & 0 &\displaystyle \frac{1}{n}\sin(n\tau) & 0 \\ 
            -2n\cos(n\tau) & 1 &\displaystyle \frac{2}{n}\left(\cos(n\tau)-1\right) &0\\ &&& \\ \hline &&& \\
            -n\sin(n\tau) & 0 & \cos(n\tau) & 0 \\
            -2n\cos(n\tau) & 0 & \displaystyle-\frac{2}{n}\left(n\sin(n\tau)+1\right) & 0 
        \end{array}\right) = \begin{pmatrix}
            \Phi_{11(t,t_0)} & \Phi_{12(t,t_0)} \\
            \Phi_{21(t,t_0)} & \Phi_{22(t,t_0)}
        \end{pmatrix} \\ \\ 
    \text{z-direction:} \\
    \Phi_{(t,t_0)} = \begin{pmatrix}
        \cos(n\tau) & \displaystyle\frac{1}{n}\sin(n\tau) \\
        -n\sin(n\tau) & \cos(n\tau)
    \end{pmatrix}
    \end{array}
\end{equation}
Where:
\begin{equation*}
    \tau = t-t_0
\end{equation*}
Desired:
\begin{equation}
    \begin{matrix}
        \begin{pmatrix}
            x1\\
            y1\\
            x2\\
            y2
        \end{pmatrix}(t_1) & = & \vec{0} \\\\
        \begin{pmatrix}
            z1\\
            z2
        \end{pmatrix}(t_1) & = & \vec{0}
    \end{matrix}
\end{equation}
% As the lecture shows, for a maneuver with $\Delta v_x = 0$, the maneuver time equals one orbital period $t_1 = T_1$. The transition matrix quadrants for $n\tau = 2\pi$:
% \begin{equation}
%     \begin{array}{lcl}
%         \Phi_{11}=\begin{pmatrix}
%             1 & 0 \\
%             -2n & 1
%         \end{pmatrix} && \Phi_{12}=\begin{pmatrix}
%             0 & 0 \\
%             0 & 0
%         \end{pmatrix} \\\\
%         \Phi_{21}=\begin{pmatrix}
%             0 & 0 \\
%             -2n & 0
%         \end{pmatrix} && \Phi_{22}=\begin{pmatrix}
%             1 & 0 \\
%             \displaystyle-\frac{2}{n} & 0
%         \end{pmatrix}
%     \end{array}
% \end{equation}
The required velocity components are found from the first 2 lines of $\vec{x}(t_1)=\Phi_(t_1,0)\vec{x(0)}$:
\begin{equation}
    \begin{pmatrix}
        x_{(t_1)}\\y_{(t_1)}
    \end{pmatrix} = \vec{0} = \Phi_{11(t_1, 0)}\begin{pmatrix}
        x(0)\\y(0)
    \end{pmatrix} + \Phi_{12(t_1,0)}\begin{pmatrix}
        \dot{x}_{req} \\
        \dot{y}_{req}
    \end{pmatrix}
\end{equation}
\begin{equation*}
    \Downarrow
\end{equation*}
\begin{equation}
    \begin{pmatrix}
        \dot{x}_{req} \\
        \dot{y}_{req}
    \end{pmatrix} = -\Phi_{12(t_1,0)}^{-1}\Phi_{11(t_1,0)}\begin{pmatrix}
        x\\y
    \end{pmatrix} \equiv \mathbf{C}^*_{(t_1,0)} \begin{pmatrix}
        x\\y
    \end{pmatrix}
\end{equation}
The terminal velocity is determined from the last two rows of $\vec{x}(t_1)=\Phi_(t_1,0)\vec{x(0)}$:
\begin{equation}
    \begin{pmatrix}
        \dot{x}_{t_1}\\\dot{y}_{t_1}
    \end{pmatrix} = \Phi_{21(t_1,0)}\begin{pmatrix}
        x\\y
    \end{pmatrix} + \Phi_{22(t_1,0)}\begin{pmatrix}
        \dot{x}_{req} \\
        \dot{y}_{req}
    \end{pmatrix}
\end{equation}
Because we can't apply a velocity pulse in the x direction, we set $\dot{x}_{t_1}$ to be zero. Using \emph{MatLab}, the first equation is:
\begin{equation}
    abc
\end{equation}



\end{document}