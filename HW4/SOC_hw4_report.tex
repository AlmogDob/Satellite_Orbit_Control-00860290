\documentclass[11pt, a4paper]{article}

\usepackage{amsmath, amssymb, titling}
\usepackage[margin=2.5cm]{geometry}
\usepackage[colorlinks=true, linkcolor=black, urlcolor=black, citecolor=black]{hyperref}
\usepackage{graphicx}
\usepackage{float}
\usepackage{fancyhdr, lastpage}
\usepackage{xcolor}

\renewcommand\maketitlehooka{\null\mbox{}\vfill}
\renewcommand\maketitlehookd{\vfill\null}

\title{Satellite Orbit Control \\ HW4}
\author{Almog Dobrescu\\\\ID 214254252}

\pagestyle{fancy}
\cfoot{Page \thepage\ of \pageref{LastPage}}

\begin{document}

\maketitle

\thispagestyle{empty}
\newpage
\setcounter{page}{1}

\tableofcontents
\vfil
\listoffigures
\newpage

\section{Given}
\begin{equation*}
    \begin{matrix}
        T_1=100\left[min\right] = 6\cdot10^3\left[sec\right] && T_2 = T_1 = 6\cdot10^3\left[sec\right] \\
        e_1 = 0 && e_2 = 0 \\
        a_1 = \sqrt[3]{\frac{\mu T_1^2}{4\pi^2}} = 7.1366\cdot10^3\left[km\right] && a_2 = a_1 = 7.1366\cdot10^3\left[km\right]
    \end{matrix}\\
\end{equation*}
\begin{equation*}
    \alpha=\Delta i = 0.01^\circ
\end{equation*}
In CW frame with origin at Satellite \#1 and at $t=0$:
\begin{equation*}
    \begin{matrix}
    \begin{pmatrix}
        x_2(0)=0 \\ y_2(0)=-1 \\ z_2(0)=1
    \end{pmatrix}\left[\mathrm{km}\right] &&
    \begin{pmatrix}
        \dot{x}_2(0)=0 \\ \dot{y}_2(0) =0 \\ \dot{z}_2(0)=-0.74267\cdot n
    \end{pmatrix}\displaystyle\left[\frac{\mathrm{km}}{\mathrm{sec}}\right]
    \end{matrix}
\end{equation*}

\subsection{Desired}
\begin{equation*}
    \begin{matrix}
    \begin{pmatrix}
        x_2(t_f)=0 \\ y_2(t_f)=0 \\ z_2(t_f)=0
    \end{pmatrix} &&
    \begin{pmatrix}
        \dot{x}_2(t_f)=0 \\ \dot{y}_2(t_f)=0 \\ \dot{z}_2(t_f)=0
    \end{pmatrix}
    \end{matrix}
\end{equation*}

\subsection{Limitations}
\begin{equation*}
    a_\text{max} = 4\cdot10^{-5}\left[\frac{\mathrm{km}}{\mathrm{sec}^2}\right]
\end{equation*}

\section{The CW equations}
\begin{equation}
    \left\{\begin{array}{l}
        \ddot{x}-2n\dot{y}-3n^2x=f_x\\
        \ddot{y}+2n\dot{x}=f_y\\
        \ddot{z}+n^2z=f_z
    \end{array}\right.
\end{equation}

\subsection{x-y}
\begin{equation}
    \vec{x}=\begin{pmatrix}
        x\\\dot{x}\\y\\\dot{y}
    \end{pmatrix}
\end{equation}
\begin{equation}
    \dot{\vec{x}}=F\vec{x}+G\vec{f}
\end{equation}
Where:
\begin{equation}
    \begin{matrix}
        F=\begin{pmatrix}
            0 & 1 & 0 & 0 \\
            3n^2 & 0 & 0 & 2n \\
            0 & 0 & 0 & 1 \\
            0 & -2n & 0 & 0
        \end{pmatrix} && G=\begin{pmatrix}
            0 & 0\\
            1 & 0\\
            0 & 0\\
            0 & 1
        \end{pmatrix} && f=\begin{pmatrix}
            f_x\\f_y
        \end{pmatrix}
    \end{matrix}
\end{equation}

\subsection{z}
\begin{equation}
    \vec{x}=\begin{pmatrix}
        z\\\dot{z}
    \end{pmatrix}
\end{equation}
\begin{equation}
    \dot{\vec{x}}=F\vec{x}+Gf
\end{equation}
Where:
\begin{equation}
    \begin{matrix}
        F=\begin{pmatrix}
            0 & 1 \\
            -n^2 & 0
        \end{pmatrix} && G=\begin{pmatrix}
            0 \\
            1
        \end{pmatrix} && f=\begin{pmatrix}
            f_z
        \end{pmatrix}
    \end{matrix}
\end{equation}

\subsection{x-y-z}
The equations of motion in state space form are therefor:
\begin{equation}
    \vec{x} = \begin{pmatrix}
        x & \dot{x} & y & \dot{y} & z & \dot{z}
    \end{pmatrix}^T
\end{equation}
\begin{equation}
    \dot{\vec{x}}=F\vec{x}+G\vec{f}
\end{equation}
Where:
\begin{equation}
    \begin{matrix}
        F=\begin{pmatrix}
            0 & 1 & 0 & 0 & 0 & 0 \\
            3n^2 & 0 & 0 & 2n & 0 & 0 \\
            0 & 0 & 0 & 1 & 0 & 0 \\
            0 & -2n & 0 & 0 & 0 & 0 \\
            0 & 0 & 0 & 0 & 0 & 1 \\
            0 & 0 & 0 & 0 & -n^2 & 0 
        \end{pmatrix} && G=\begin{pmatrix}
            0 & 0 & 0\\
            1 & 0 & 0\\
            0 & 0 & 0\\
            0 & 1 & 0\\
            0 & 0 & 0\\
            0 & 0 & 1
        \end{pmatrix} && f=\begin{pmatrix}
            f_x\\f_y\\f_z
        \end{pmatrix}
    \end{matrix}
\end{equation}
\newpage

\section{A and B}
The desired poles are given by the following equation:
\begin{equation}
    P = 10\cdot\begin{pmatrix}
        -n+i\cdot \frac{n}{600}\\-n-i\cdot \frac{n}{600}\\
        -4n+i\cdot 3\frac{n}{600}\\-4n-i\cdot 3\frac{n}{600}\\
        -3n+i\cdot \frac{n}{600}\\-3n-i\cdot \frac{n}{600}
    \end{pmatrix} = \begin{pmatrix}
        -0.01047+i\cdot0.00001745\\ -0.01047-i\cdot0.00001745\\
        -0.04189+i\cdot0.00005236\\ -0.04189-i\cdot0.00005236\\
        -0.03142+i\cdot0.00001745\\ -0.03142-i\cdot0.00001745
    \end{pmatrix}
\end{equation}
By using the function \emph{place} in Matlab, we get:
\begin{equation}
    K = \begin{pmatrix}
        4.406\cdot10^{-4}&0.0523&1.6865\cdot10^{-6}&0.0021&-1.1211\cdot10^{-5}&-3.0579\cdot10^{-4}\\
        -1.0725\cdot10^{-6}&-0.0021&3.3014\cdot10^{-4}&0.0419&1.5749\cdot10^{-6}&3.7109\cdot10^{-5}\\
        -9.2809\cdot10^{-6}&-2.0798\cdot10^{-4}&2.6270\cdot10^{-6}&7.6995\cdot10^{-5}&0.0013&0.0733
    \end{pmatrix}
\end{equation}
We can see that $$\displaystyle\tau = \frac{1}{\mathbb{R}e\left\{P_i\right\}} < \frac{1}{10n} = \frac{1}{0.01047} = 95.4923 < 2000$$
\begin{figure}[H]
    \centering
    \includegraphics[width=1\textwidth]{images/graph1.png}
    \caption{3D figure of the orbit trajectory - AB}
    \label{fig:3D-plot-AB}
\end{figure}
\begin{figure}[H]
    \centering
    \includegraphics[width=1\textwidth]{images/graph2.png}
    \caption{2D figure of the position over time - AB}
    \label{fig:2D-plot_over_time-AB}
\end{figure}
\begin{figure}[H]
    \centering
    \includegraphics[width=1\textwidth]{images/graph3.png}
    \caption{Thrust acceleration components and total thrust acceleration over time - AB}
    \label{fig:accel_over_time-AB}
\end{figure}
\begin{figure}[H]
    \centering
    \includegraphics[width=1\textwidth]{images/graph4.png}
    \caption{Total $\Delta v$ over time - AB}
    \label{fig:delta_v_over_time-AB}
\end{figure}
We can see that we indeed acomplished the design criteria:
\begin{itemize}
    \item The thrust doesn't exceed the maximum available thrust.
    \item The miss distance at the final desired time is less than $1[\mathrm{m}]\ \left(2.3893\cdot10^{-5}\left[\mathrm{m}\right]\right)$.
    \item The miss velocity at the final desired time is less than $1\left[\displaystyle\frac{\mathrm{cm}}{\mathrm{sec}}\right]\ \left(2.5057\cdot10^{-5}\left[\displaystyle\frac{\mathrm{cm}}{\mathrm{sec}}\right]\right)$.
\end{itemize}

The total $\Delta v$ is: $0.021596\left[\displaystyle\frac{\mathrm{km}}{\mathrm{sec}}\right]$
\newpage

\section{C}
For all $t>500\left[\mathrm{sec}\right]$, the measered position at z is zero.
\begin{figure}[H]
    \centering
    \includegraphics[width=1\textwidth]{images/graph5.png}
    \caption{3D figure of the orbit trajectory - C}
    \label{fig:3D-plot-caseC}
\end{figure}
\begin{figure}[H]
    \centering
    \includegraphics[width=1\textwidth]{images/graph6.png}
    \caption{2D figure of the position over time - C}
    \label{fig:2D-plot_over_time-caseC}
\end{figure}
\begin{figure}[H]
    \centering
    \includegraphics[width=0.75\textwidth]{images/graph7.png}
    \caption{Thrust acceleration components and total thrust acceleration over time - C}
    \label{fig:accel_over_time-caseC}
\end{figure}
\begin{figure}[H]
    \centering
    \includegraphics[width=0.75\textwidth]{images/graph8.png}
    \caption{Total $\Delta v$ over time - C}
    \label{fig:delta_v_over_time-caseC}
\end{figure}
We can see that we indeed acomplished the design criteria:
\begin{itemize}
    \item The thrust doesn't exceed the maximum available thrust.
    \item The miss distance at the final desired time is less than $1[\mathrm{m}]\ \left(0.895205\left[\mathrm{m}\right]\right)$.
    \item The miss velocity at the final desired time is less than $1\left[\displaystyle\frac{\mathrm{cm}}{\mathrm{sec}}\right]\ \left(2.50828\cdot10^{-5}\left[\displaystyle\frac{\mathrm{cm}}{\mathrm{sec}}\right]\right)$.
\end{itemize}

we can see that unlike the normal operation, the thrust doesn't go to zero but to a constatn and the $\Delta v$ grows a bit. Because I chose good poles we still acomplished the design criteria. We can see that because we lost control on the z direction, the miss distance increased a lot and the miss velocity increased a bit



\end{document}