\documentclass[11pt, a4paper]{article}

\usepackage{amsmath, amssymb, titling}
\usepackage[margin=2.75cm]{geometry}
\usepackage[colorlinks=true, linkcolor=black, urlcolor=black, citecolor=black]{hyperref}
\usepackage{graphicx}
\usepackage{float}
\usepackage{fancyhdr, lastpage}
\usepackage{xcolor}

\renewcommand\maketitlehooka{\null\mbox{}\vfill}
\renewcommand\maketitlehookd{\vfill\null}

\title{Satellite Orbit Control \\ HW4}
\author{Almog Dobrescu\\\\ID 214254252}

\pagestyle{fancy}
\cfoot{Page \thepage\ of \pageref{LastPage}}

\begin{document}

\maketitle

\thispagestyle{empty}
\newpage
\setcounter{page}{1}

\tableofcontents
\vfil
\listoffigures
\newpage

\section{Given}
\begin{equation*}
    \begin{matrix}
        T_1=100\left[min\right] = 6\cdot10^3\left[sec\right] && T_2 = T_1 = 6\cdot10^3\left[sec\right] \\
        e_1 = 0 && e_2 = 0 \\
        a_1 = \sqrt[3]{\frac{\mu T_1^2}{4\pi^2}} = 7.1366\cdot10^3\left[km\right] && a_2 = a_1 = 7.1366\cdot10^3\left[km\right]
    \end{matrix}\\
\end{equation*}
\begin{equation*}
    \alpha=\Delta i = 0.01^\circ
\end{equation*}
In CW frame with origin at Satellite \#1 and at $t=0$:
\begin{equation*}
    \begin{matrix}
    \begin{pmatrix}
        x_2(0)=0 \\ y_2(0)=-1 \\ z_2(0)=1
    \end{pmatrix}\left[\mathrm{km}\right] &&
    \begin{pmatrix}
        \dot{x}_2(0)=0 \\ \dot{y}_2(0) =0 \\ \dot{z}_2(0)=-0.74267\cdot n
    \end{pmatrix}\displaystyle\left[\frac{\mathrm{km}}{\mathrm{sec}}\right]
    \end{matrix}
\end{equation*}

\subsection{Desired}
\begin{equation*}
    \begin{matrix}
    \begin{pmatrix}
        x_2(t_f)=0 \\ y_2(t_f)=0 \\ z_2(t_f)=0
    \end{pmatrix} &&
    \begin{pmatrix}
        \dot{x}_2(t_f)=0 \\ \dot{y}_2(t_f)=0 \\ \dot{z}_2(t_f)=0
    \end{pmatrix}
    \end{matrix}
\end{equation*}

\subsection{Limitations}
% The engine can't create thrust in the \emph{x} direation and:
\begin{equation*}
    a_\text{max} = 4\cdot10^{-5}\left[\frac{\mathrm{km}}{\mathrm{sec}^2}\right]
\end{equation*}

\section{The CW equations}
\begin{equation}
    \left\{\begin{array}{l}
        \ddot{x}-2n\dot{y}-3n^2x=f_x\\
        \ddot{y}+2n\dot{x}=f_y\\
        \ddot{z}+n^2z=f_z
    \end{array}\right.
\end{equation}

\subsection{x-y}
\begin{equation}
    \vec{x}=\begin{pmatrix}
        x\\\dot{x}\\y\\\dot{y}
    \end{pmatrix}
\end{equation}
\begin{equation}
    \dot{\vec{x}}=F\vec{x}+G\vec{f}
\end{equation}
Where:
\begin{equation}
    \begin{matrix}
        F=\begin{pmatrix}
            0 & 1 & 0 & 0 \\
            3n^2 & 0 & 0 & 2n \\
            0 & 0 & 0 & 1 \\
            0 & -2n & 0 & 0
        \end{pmatrix} && G=\begin{pmatrix}
            0 & 0\\
            1 & 0\\
            0 & 0\\
            0 & 1
        \end{pmatrix} && f=\begin{pmatrix}
            f_x\\f_y
        \end{pmatrix}
    \end{matrix}
\end{equation}

\subsection{z}
\begin{equation}
    \vec{x}=\begin{pmatrix}
        z\\\dot{z}
    \end{pmatrix}
\end{equation}
\begin{equation}
    \dot{\vec{x}}=F\vec{x}+Gf
\end{equation}
Where:
\begin{equation}
    \begin{matrix}
        F=\begin{pmatrix}
            0 & 1 \\
            -n^2 & 0
        \end{pmatrix} && G=\begin{pmatrix}
            0 \\
            1
        \end{pmatrix} && f=\begin{pmatrix}
            f_z
        \end{pmatrix}
    \end{matrix}
\end{equation}

\subsection{x-y-z}
The equations of motion in state space form are therefor:
\begin{equation}
    \vec{x} = \begin{pmatrix}
        x & \dot{x} & y & \dot{y} & z & \dot{z}
    \end{pmatrix}^T
\end{equation}
\begin{equation}
    \dot{\vec{x}}=F\vec{x}+G\vec{f}
\end{equation}
Where:
\begin{equation}
    \begin{matrix}
        F=\begin{pmatrix}
            0 & 1 & 0 & 0 & 0 & 0 \\
            3n^2 & 0 & 0 & 2n & 0 & 0 \\
            0 & 0 & 0 & 1 & 0 & 0 \\
            0 & -2n & 0 & 0 & 0 & 0 \\
            0 & 0 & 0 & 0 & 0 & 1 \\
            0 & 0 & 0 & 0 & -n^2 & 0 
        \end{pmatrix} && G=\begin{pmatrix}
            0 & 0 & 0\\
            1 & 0 & 0\\
            0 & 0 & 0\\
            0 & 1 & 0\\
            0 & 0 & 0\\
            0 & 0 & 1
        \end{pmatrix} && f=\begin{pmatrix}
            f_x\\f_y\\f_z
        \end{pmatrix}
    \end{matrix}
\end{equation}
\subsection{The homogeneous solution}
\begin{equation}
    \vec{x} = \begin{pmatrix}
        x & y & z & \dot{x} & \dot{y} & \dot{z}
    \end{pmatrix}^T
\end{equation}
\begin{equation}
    \begin{array}{l}
        \text{x-y plane:} \\
        \Phi_{\left\{t,t_0\right\}} = \left(\begin{array}{cc|cc}
            4-3\cos(n\tau) & 0 &\displaystyle \frac{1}{n}\sin(n\tau) &\displaystyle \frac{2}{n}\left(1-\cos(n\tau)\right) \\ 
            6\left(\sin(n\tau)-n\tau\right) & 1 &\displaystyle \frac{2}{n}\left(\cos(n\tau)-1\right) &\displaystyle \frac{1}{n}\left(4\sin(n\tau)-3n\tau\right)\\ &&& \\ \hline &&& \\
            3n\sin(n\tau) & 0 & \cos(n\tau) & 2\sin(n\tau) \\
            6n\left(\cos(n\tau)-1\right) & 0 & -2\sin(n\tau) & 4\cos(n\tau)-3 
        \end{array}\right) = \begin{pmatrix}
            \Phi_{11(t,t_0)} & \Phi_{12(t,t_0)} \\
            \Phi_{21(t,t_0)} & \Phi_{22(t,t_0)}
        \end{pmatrix} \\ \\
    \text{z-direction:} \\
    \Phi_{(t,t_0)} = \begin{pmatrix}
        \cos(n\tau) & \displaystyle\frac{1}{n}\sin(n\tau) \\
        -n\sin(n\tau) & \cos(n\tau)
    \end{pmatrix}
    \end{array}
\end{equation}
Where:
\begin{equation*}
    \tau = t-t_0
\end{equation*}
So the full homogeneous solution in state space from:
\begin{equation*}
    \Phi_{(t,t_0)} = \left(\begin{array}{ccc|ccc}
        4-3\cos(n\tau) & 0 & 0 &\displaystyle \frac{1}{n}\sin(n\tau) &\displaystyle \frac{2}{n}\left(1-\cos(n\tau)\right) & 0 \\ 
        6\left(\sin(n\tau)-n\tau\right) & 1 & 0 &\displaystyle \frac{2}{n}\left(\cos(n\tau)-1\right) &\displaystyle \frac{1}{n}\left(4\sin(n\tau)-3n\tau\right) & 0 \\ 0 & 0 & \cos(n\tau) & 0 & 0 & \displaystyle\frac{1}{n}\sin(n\tau) \\ &&& \\ \hline &&& \\
        3n\sin(n\tau) & 0 & 0 & \cos(n\tau) & 2\sin(n\tau) & 0 \\
        6n\left(\cos(n\tau)-1\right) & 0 & 0 & -2\sin(n\tau) & 4\cos(n\tau)-3 & 0 \\
        0 & 0 & -n\sin(n\tau) & 0 & 0 & \cos(n\tau) 
    \end{array}\right) =  
\end{equation*}
\begin{equation}
    \begin{pmatrix}
        \Phi_{11(t,t_0)} & \Phi_{12(t,t_0)} \\
        \Phi_{21(t,t_0)} & \Phi_{22(t,t_0)}
    \end{pmatrix} 
\end{equation}
\newpage

\section{A}
The desired poles are given by the following equation:
\begin{equation}
    P = 10\cdot\begin{pmatrix}
        -n+i\cdot n\\-n-i\cdot n\\
        -4n+i\cdot 3n\\-4n-i\cdot 3n\\
        -3n+i\cdot n\\-3n-i\cdot n
    \end{pmatrix} = \begin{pmatrix}
        -0.0105+i\cdot0.0105\\ -0.0105-i\cdot0.0105\\
        -0.0419+i\cdot0.0314\\ -0.0419-i\cdot0.0314\\
        -0.0314+i\cdot0.0105\\ -0.0314-i\cdot0.0105
    \end{pmatrix}
\end{equation}
By using the function \emph{place} in Matlab, we get:
\begin{equation}
    K = \begin{pmatrix}
        0.0013&0.0623&-3.0786\cdot10^{-5}&0.0122&-9.0833\cdot10^{-4}&-0.0308\\
        1.8104\cdot10^{-4}&-0.0121&6.1079\cdot10^{-4}&0.0464&-3.0853\cdot10^{-5}&-0.0077\\
        -2.4680\cdot10^{-4}&0.0059&-3.4367\cdot10^{-4}&-0.0033&9.6612\cdot10^{-4}&0.0588
    \end{pmatrix}
\end{equation}
We can see that $$\displaystyle\tau = \frac{1}{\mathbb{R}e\left\{P_i\right\}} < \frac{1}{10n} = \frac{1}{0.0105} = 95.4930 < 2000$$



\end{document}