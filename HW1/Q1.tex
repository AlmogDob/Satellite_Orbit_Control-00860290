Given:
Because the ascending node position is given in ECI, $ r_{an, 3} = 0$ so: 
\begin{equation}
    \vec{r}_{an} = \begin{pmatrix}
            5000\\8500\\0
        \end{pmatrix} [km] \hspace{0.25cm} r = 9.8615 \cdot 10^3 [km]
\end{equation}
\\
From the period:
\begin{equation}
    T = \frac{2\pi}{\sqrt{\frac{\mu}{a^3}}}
    \Rightarrow
    a = \sqrt[3]{\frac{\mu T^2}{4\pi ^2}}
    \Rightarrow
    \colorbox{yellow}{$a = 8.5007 \cdot 10^3$}
\end{equation}
\\
The velocity at the ascending node is:
\begin{equation}
    v = \sqrt{2\mu \left(\frac{1}{r} - \frac{1}{2a}\right)}
    \Rightarrow
    v = 5.8266 \left[\frac{km}{sec}\right] 
\end{equation}
\begin{equation}
    v = \sqrt{2.15^2 + 2.05^2 + v_{an, 3}^2} 
    \Rightarrow
    v_{an, 3} = \pm 5.0124 \left[\frac{km}{sec}\right] 
\end{equation}
Find the angular momentum in order to find the ascending node line:
\begin{equation}
    \vec{h} = \vec{r} \times \vec{v} = \begin{vmatrix}
        i & j & k \\ 5000 & 8500 & 0 \\ 2.15 & -2.05 & \pm 5.0124
    \end{vmatrix} = \begin{pmatrix}
        \pm 42605.4 \\ \mp 25062 \\ -28525
    \end{pmatrix}
\end{equation}
\begin{equation}
    \vec{n} = \vec{z} \times \vec{h} = -h_y\vec{x}+h_x\vec{y} = \begin{pmatrix}
        \pm 25062 \\ \pm 42605.4 \\ 0
    \end{pmatrix}
    \hspace{0.25cm} n = 4.9430\cdot 10^4
\end{equation}
\\
The radius at the ascending node is parallel to the ascending node line which means that they have the same sign in every direction:
\begin{equation}
    v_{an, 3} = 5.0124 \left[\frac{km}{sec}\right] 
    \Rightarrow
    \vec{v}_{an} = \begin{pmatrix}
        2.15\\-2.05\\5.0124
    \end{pmatrix} \left[\frac{km}{sec}\right] 
\end{equation}
\begin{equation}
    \vec{h} = \begin{pmatrix}
        42605.4 \\ -25062 \\ -28525
    \end{pmatrix}\left[\frac{km^2}{sec}\right] \hspace{0.25cm} h = 5.7070\cdot 10^4 \left[\frac{km^2}{sec}\right]
\end{equation}
\\
The eccentricity is given by:
\begin{equation}
    \vec{e} = \frac{\vec{v}\times\vec{h}}{\mu} - \frac{\vec{r}}{r} = \begin{pmatrix}
        -0.0452 \\ -0.1723 \\ 0.0839
    \end{pmatrix}
    \hspace{0.25cm} 
    \colorbox{yellow}{$e = 0.1969$}    
\end{equation}
\\
The angle of inclination $(0^\circ\leq i \leq 180^\circ)$:
\begin{equation}
    \cos (i) = \frac{h_z}{h} = \frac{-28525}{57070}
    \Rightarrow
    \colorbox{yellow}{$i = 2.0942[rad] = 119.9888^\circ$}
\end{equation}
\\
Argument of perigee:
\begin{equation}
    \cos (\omega) = \frac{\vec{n} \cdot \vec{e}}{n\cdot e},\ \  \sin (\omega) = \sqrt{1-\cos^2(\omega)}
\end{equation}
$$\Downarrow$$
\begin{equation}
    \omega = \text{atan2}\left[\text{sign}(e_z)\cdot\sqrt{1-\left(\frac{\vec{n}\cdot\vec{e}}{n\cdot e}\right)^2}, \frac{\vec{n}\cdot\vec{e}}{n\cdot e}\right]
\end{equation}
$$\Downarrow$$
\begin{align}
\colorbox{yellow}{$\omega = 2.6270 [rad] = 150.5160^\circ$}
\end{align}
Longitude of ascending node:
\begin{equation}
    \cos(\Omega) = \frac{n_x}{n},\ \ \sin(\Omega) = \frac{n_y}{n}\Rightarrow\Omega = \text{atan2}\left(\frac{n_y}{n}, \frac{n_x}{n}\right)
\end{equation}
$$\Downarrow$$
\begin{align}
\colorbox{yellow}{$\Omega = 1.0391 [rad] = 59.5303^\circ$}
\end{align}

