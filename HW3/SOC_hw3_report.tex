\documentclass[11pt, a4paper]{article}

\usepackage{amsmath, amssymb, titling}
\usepackage[margin=2.75cm]{geometry}
\usepackage[colorlinks=true, linkcolor=black, urlcolor=black, citecolor=black]{hyperref}
\usepackage{graphicx}
\usepackage{float}
\usepackage{fancyhdr, lastpage}
\usepackage{xcolor}

\renewcommand\maketitlehooka{\null\mbox{}\vfill}
\renewcommand\maketitlehookd{\vfill\null}

\title{Satellite Orbit Control \\ HW3}
\author{Almog Dobrescu\\\\ID 214254252}

\pagestyle{fancy}
\cfoot{Page \thepage\ of \pageref{LastPage}}

\begin{document}

\maketitle

\thispagestyle{empty}
\newpage
\setcounter{page}{1}

\tableofcontents
\vfil
\listoffigures
\newpage

\section{Given}
\begin{equation*}
    \begin{matrix}
        T_1=100\left[min\right] = 6\cdot10^3\left[sec\right] && T_2 = T_1 = 6\cdot10^3\left[sec\right] \\
        e_1 = 0 && e_2 = 0 \\
        a_1 = \sqrt[3]{\frac{\mu T_1^2}{4\pi^2}} = 7.1366\cdot10^3\left[km\right] && a_2 = a_1 = 7.1366\cdot10^3\left[km\right]
    \end{matrix}\\
\end{equation*}
\begin{equation*}
    \alpha=\Delta i = 0.01^\circ
\end{equation*}
In CW frame with origin at Satellite \#1 and at $t=0$:
\begin{equation*}
    \begin{matrix}
    \begin{pmatrix}
        x_2(0)=0 \\ y_2(0)=-1 \\ z_2(0)=1
    \end{pmatrix}\left[\mathrm{km}\right] &&
    \begin{pmatrix}
        \dot{x}_2(0)=0 \\ \dot{y}_2(0) =0 \\ \dot{z}_2(0)=-0.74267\cdot n
    \end{pmatrix}\displaystyle\left[\frac{\mathrm{km}}{\mathrm{sec}}\right]
    \end{matrix}
\end{equation*}

\subsection{Desired}
\begin{equation*}
    \begin{matrix}
    \begin{pmatrix}
        x_2(t_f)=0 \\ y_2(t_f)=0 \\ z_2(t_f)=0
    \end{pmatrix} &&
    \begin{pmatrix}
        \dot{x}_2(t_f)=0 \\ \dot{y}_2(t_f)=0 \\ \dot{z}_2(t_f)=0
    \end{pmatrix}
    \end{matrix}
\end{equation*}

\subsection{Limitations}
% The engine can't create thrust in the \emph{x} direation and:
\begin{equation*}
    a_\text{max} = 8\cdot10^{-6}\left[\frac{\mathrm{km}}{\mathrm{sec}^2}\right]
\end{equation*}

\section{The CW equations}
\begin{equation}
    \left\{\begin{array}{l}
        \ddot{x}-2n\dot{y}-3n^2x=f_x\\
        \ddot{y}+2n\dot{x}=f_y\\
        \ddot{z}+n^2z=f_z
    \end{array}\right.
\end{equation}

\subsection{x-y}
\begin{equation}
    \vec{x}=\begin{pmatrix}
        x\\\dot{x}\\y\\\dot{y}
    \end{pmatrix}
\end{equation}
\begin{equation}
    \dot{\vec{x}}=F\vec{x}+G\vec{f}
\end{equation}
Where:
\begin{equation}
    \begin{matrix}
        F=\begin{pmatrix}
            0 & 1 & 0 & 0 \\
            3n^2 & 0 & 0 & 2n \\
            0 & 0 & 0 & 1 \\
            0 & -2n & 0 & 0
        \end{pmatrix} && G=\begin{pmatrix}
            0 & 0\\
            1 & 0\\
            0 & 0\\
            0 & 1
        \end{pmatrix} && f=\begin{pmatrix}
            f_x\\f_y
        \end{pmatrix}
    \end{matrix}
\end{equation}

\subsection{z}
\begin{equation}
    \vec{x}=\begin{pmatrix}
        z\\\dot{z}
    \end{pmatrix}
\end{equation}
\begin{equation}
    \dot{\vec{x}}=F\vec{x}+Gf
\end{equation}
Where:
\begin{equation}
    \begin{matrix}
        F=\begin{pmatrix}
            0 & 1 \\
            -n^2 & 0
        \end{pmatrix} && G=\begin{pmatrix}
            0 \\
            1
        \end{pmatrix} && f=\begin{pmatrix}
            f_z
        \end{pmatrix}
    \end{matrix}
\end{equation}

\subsection{x-y-z}
The equations of motion in state space form are therefor:
\begin{equation}
    \vec{x} = \begin{pmatrix}
        x & \dot{x} & y & \dot{y} & z & \dot{z}
    \end{pmatrix}^T
\end{equation}
\begin{equation}
    \dot{\vec{x}}=F\vec{x}+G\vec{f}
\end{equation}
Where:
\begin{equation}
    \begin{matrix}
        F=\begin{pmatrix}
            0 & 1 & 0 & 0 & 0 & 0 \\
            3n^2 & 0 & 0 & 2n & 0 & 0 \\
            0 & 0 & 0 & 1 & 0 & 0 \\
            0 & -2n & 0 & 0 & 0 & 0 \\
            0 & 0 & 0 & 0 & 0 & 1 \\
            0 & 0 & 0 & 0 & -n^2 & 0 
        \end{pmatrix} && G=\begin{pmatrix}
            0 & 0 & 0\\
            1 & 0 & 0\\
            0 & 0 & 0\\
            0 & 1 & 0\\
            0 & 0 & 0\\
            0 & 0 & 1
        \end{pmatrix} && f=\begin{pmatrix}
            f_x\\f_y\\f_z
        \end{pmatrix}
    \end{matrix}
\end{equation}
\subsection{The homogeneous solution}
\begin{equation}
    \vec{x} = \begin{pmatrix}
        x & y & z & \dot{x} & \dot{y} & \dot{z}
    \end{pmatrix}^T
\end{equation}
\begin{equation}
    \begin{array}{l}
        \text{x-y plane:} \\
        \Phi_{\left\{t,t_0\right\}} = \left(\begin{array}{cc|cc}
            4-3\cos(n\tau) & 0 &\displaystyle \frac{1}{n}\sin(n\tau) &\displaystyle \frac{2}{n}\left(1-\cos(n\tau)\right) \\ 
            6\left(\sin(n\tau)-n\tau\right) & 1 &\displaystyle \frac{2}{n}\left(\cos(n\tau)-1\right) &\displaystyle \frac{1}{n}\left(4\sin(n\tau)-3n\tau\right)\\ &&& \\ \hline &&& \\
            3n\sin(n\tau) & 0 & \cos(n\tau) & 2\sin(n\tau) \\
            6n\left(\cos(n\tau)-1\right) & 0 & -2\sin(n\tau) & 4\cos(n\tau)-3 
        \end{array}\right) = \begin{pmatrix}
            \Phi_{11(t,t_0)} & \Phi_{12(t,t_0)} \\
            \Phi_{21(t,t_0)} & \Phi_{22(t,t_0)}
        \end{pmatrix} \\ \\
    \text{z-direction:} \\
    \Phi_{(t,t_0)} = \begin{pmatrix}
        \cos(n\tau) & \displaystyle\frac{1}{n}\sin(n\tau) \\
        -n\sin(n\tau) & \cos(n\tau)
    \end{pmatrix}
    \end{array}
\end{equation}
Where:
\begin{equation*}
    \tau = t-t_0
\end{equation*}
So the full homogeneous solution in state space from:
\begin{equation*}
    \Phi_{(t,t_0)} = \left(\begin{array}{ccc|ccc}
        4-3\cos(n\tau) & 0 & 0 &\displaystyle \frac{1}{n}\sin(n\tau) &\displaystyle \frac{2}{n}\left(1-\cos(n\tau)\right) & 0 \\ 
        6\left(\sin(n\tau)-n\tau\right) & 1 & 0 &\displaystyle \frac{2}{n}\left(\cos(n\tau)-1\right) &\displaystyle \frac{1}{n}\left(4\sin(n\tau)-3n\tau\right) & 0 \\ 0 & 0 & \cos(n\tau) & 0 & 0 & \displaystyle\frac{1}{n}\sin(n\tau) \\ &&& \\ \hline &&& \\
        3n\sin(n\tau) & 0 & 0 & \cos(n\tau) & 2\sin(n\tau) & 0 \\
        6n\left(\cos(n\tau)-1\right) & 0 & 0 & -2\sin(n\tau) & 4\cos(n\tau)-3 & 0 \\
        0 & 0 & -n\sin(n\tau) & 0 & 0 & \cos(n\tau) 
    \end{array}\right) =  
\end{equation*}
\begin{equation}
    \begin{pmatrix}
        \Phi_{11(t,t_0)} & \Phi_{12(t,t_0)} \\
        \Phi_{21(t,t_0)} & \Phi_{22(t,t_0)}
    \end{pmatrix} 
\end{equation}

% \begin{equation}
%     \begin{matrix}
%         \text{x-y plane:} && \text{z-direction:} \\
%         \begin{matrix}
%             x1 = x \\
%             x2 = \dot{x}
%         \end{matrix} && z1 = z \\
%         \begin{matrix}
%             y1 = y \\
%             y2 = \dot{y}
%         \end{matrix} && z2 = \dot{z}
%     \end{matrix}
% \end{equation}
% The homogeneous solution:
% \begin{equation}
%     \vec{x}(t) = \Phi_{(t,t_0)}\vec{x}_0
% \end{equation}
% \begin{equation}
%     \begin{array}{l}
%         \text{x-y plane:} \\
%         \Phi_{(t,t_0)} = \left(\begin{array}{cc|cc}
%             4-3\cos(n\tau) & 0 &\displaystyle \frac{1}{n}\sin(n\tau) &\displaystyle \frac{2}{n}\left(1-\cos(n\tau)\right) \\ 
%             6\left(\sin(n\tau)-n\tau\right) & 1 &\displaystyle \frac{2}{n}\left(\cos(n\tau)-1\right) &\displaystyle \frac{1}{n}\left(4\sin(n\tau)-3n\tau\right)\\ &&& \\ \hline &&& \\
%             3n\sin(n\tau) & 0 & \cos(n\tau) & 2\sin(n\tau) \\
%             6n\left(\cos(n\tau)-1\right) & 0 & -2\sin(n\tau) & 4\cos(n\tau)-3 
%         % \Phi_{(t,t_0)} = \left(\begin{array}{cc|cc}
%         %     \cos(n\tau) & 0 &\displaystyle \frac{1}{n}\sin(n\tau) & 0 \\ 
%         %     -2n\cos(n\tau) & 1 &\displaystyle \frac{2}{n}\left(\cos(n\tau)-1\right) &0\\ &&& \\ \hline &&& \\
%         %     -n\sin(n\tau) & 0 & \cos(n\tau) & 0 \\
%         %     -2n\cos(n\tau) & 0 & \displaystyle-\frac{2}{n}\left(n\sin(n\tau)+1\right) & 0 
%         \end{array}\right) = \begin{pmatrix}
%             \Phi_{11(t,t_0)} & \Phi_{12(t,t_0)} \\
%             \Phi_{21(t,t_0)} & \Phi_{22(t,t_0)}
%         \end{pmatrix} \\ \\ 
%     \text{z-direction:} \\
%     \Phi_{(t,t_0)} = \begin{pmatrix}
%         \cos(n\tau) & \displaystyle\frac{1}{n}\sin(n\tau) \\
%         -n\sin(n\tau) & \cos(n\tau)
%     \end{pmatrix}
%     \end{array}
% \end{equation}
% Where:
% \begin{equation*}
%     \tau = t-t_0
% \end{equation*}
% Desired:
% \begin{equation}
%     \begin{matrix}
%         \begin{pmatrix}
%             x1\\
%             y1\\
%             x2\\
%             y2
%         \end{pmatrix}(t_f) & = & \vec{0} \\\\
%         \begin{pmatrix}
%             z1\\
%             z2
%         \end{pmatrix}(t_f) & = & \vec{0}
%     \end{matrix}
% \end{equation}

\newpage
\section{case A}
The desired maneuver time is $2000\left[\mathrm{sec}\right]$. The desired approach trajectory is a straight line with constant velocity from the initial point to the origin. \\
From geometric considerations, the required velocity on the straight line is:

\begin{equation}
    \colorbox{yellow}{$\dot{\vec{x}}_{req}$} = \begin{pmatrix}
        \displaystyle\frac{x_f-x_0}{t_f}\\\\
        \displaystyle\frac{y_f-y_0}{t_f}\\\\
        \displaystyle\frac{z_f-z_0}{t_f}
    \end{pmatrix}\colorbox{yellow}{$ = \begin{pmatrix}
       0\\5\\-5
    \end{pmatrix}\cdot10^{-4}\left[\frac{\mathrm{km}}{\mathrm{sec}^2}\right]$}
\end{equation}
Which mean that the initial pulse:
\begin{equation}
    \colorbox{yellow}{$\Delta v_1 = \dot{\vec{x}}_{req} - \dot{\vec{x}}_{0} = \begin{pmatrix}
        0\\0.5\\0.2777
    \end{pmatrix}\cdot10^{-3}\left[\frac{\mathrm{km}}{\mathrm{sec}^2}\right]$}
\end{equation}
and the final pulse:
\begin{equation}
    \colorbox{yellow}{$\Delta v_f = \dot{\vec{x}}_{f} - \dot{\vec{x}}_{req} = \begin{pmatrix}
        0\\-5\\5
    \end{pmatrix}\cdot10^{-4}\left[\frac{\mathrm{km}}{\mathrm{sec}^2}\right]$}
\end{equation}



% The target trajectory:
% \begin{equation}
%     \dot{\vec{x}}_r = \underbrace{\begin{pmatrix}
%         0 & 1 & 0 & 0 & 0 & 0\\
%         0 & 0 & 0 & 0 & 0 & 0\\
%         0 & 0 & 0 & 1 & 0 & 0\\
%         0 & 0 & 0 & 0 & 0 & 0\\
%         0 & 0 & 0 & 0 & 0 & 1\\
%         0 & 0 & 0 & 0 & 0 & 0
%     \end{pmatrix}}_{\displaystyle F_r}\vec{x}_r \end{equation}
% The initial position for the approach trajectory:
% \begin{equation}
%     \vec{x}_{r(0)} = \begin{pmatrix}
%         0\\-1\\1\\
%         \displaystyle\frac{x_f-x_0}{t_f}\\
%         \displaystyle\frac{y_f-y_0}{t_f}\\
%         \displaystyle\frac{z_f-z_0}{t_f}
%     \end{pmatrix} = \begin{pmatrix}
%         0\\-1\\1\\0\\5\cdot10^{-4}\\-5\cdot10^{-4}
%     \end{pmatrix}
% \end{equation}

\noindent Since the velocity is constant:
\begin{equation*}
    \begin{pmatrix}
        \ddot{x}\\
        \ddot{y}\\
        \ddot{z}
    \end{pmatrix} = \vec{0}
\end{equation*}
By substituting the constraints in the CW equations we get:
\begin{equation}
    \left\{\begin{array}{l}
        0-2n\dot{y}=f_x\\
        0=f_y\\
        0+n^2z=f_z
    \end{array}\right.
\end{equation}
\begin{equation}
    \colorbox{yellow}{$\vec{a} = \vec{f} = \begin{pmatrix}
        -2n\cdot0.5\cdot10^{-3}\\
        0\\
        n^2z
    \end{pmatrix}$}
\end{equation}
Check that the maximum acceleration hasn't been reached:
\begin{equation}
    \begin{array}{c}
        |\vec{a}| = \sqrt{\left(2n\cdot0.5\cdot10^{-3}\right) + \left(n^2z\right)} < \sqrt{\left(2n\cdot0.5\cdot10^{-3}\right) + \left(n^2\cdot1\right)} \\\\
        |\vec{a}| < 1.5163\cdot10^{-6} < 8\cdot10^{-6} \text{\hspace{1cm}\LARGE$\surd$}
    \end{array}
\end{equation}
By solving this system numerically, we obtain the approach trajectory.
\begin{figure}[H]
    \centering
    \includegraphics[width=1\textwidth]{images/graph1.png}
    \caption{3D figure of the orbit trajectory - case A}
    \label{fig:3D-plot-caseA}
\end{figure}
\begin{figure}[H]
    \centering
    \includegraphics[width=1\textwidth]{images/graph2.png}
    \caption{2D figure of the position over time - case A}
    \label{fig:2D-plot_over_time-caseA}
\end{figure}
\begin{figure}[H]
    \centering
    \includegraphics[width=0.9\textwidth]{images/graph3.png}
    \caption{Thrust acceleration components and total thrust acceleration over time - case A}
    \label{fig:accel_over_time-caseA}
\end{figure}
\begin{figure}[H]
    \centering
    \includegraphics[width=0.9\textwidth]{images/graph4.png}
    \caption{Total $\Delta v$ over time - case A}
    \label{fig:delta_v_over_time-caseA}
\end{figure}
We can see that we indeed acomplished the design criteria:
\begin{itemize}
    \item The thrust doesn't exceed the maximum available thrust.
    \item The miss distance at the final desired time is less than $1[\mathrm{m}]\ \left(2.35517\cdot10^{-13}\left[\mathrm{m}\right]\right)$.
\end{itemize}

\newpage
\section{case B}
The velocity to be gained:
\begin{equation}
    \vec{v}_g=\vec{v}_r-\vec{v}
\end{equation}
The equation of motion:
\begin{equation}
    \frac{d\vec{v}}{dt} = \vec{g}_{\left(\vec{r}\right)}+\vec{a}_T
\end{equation}
The equation of motion for the velocity to be gained:
\begin{equation}
    \frac{d\vec{v}_g}{dt} = \frac{d\vec{v}_r}{dt} - \frac{d\vec{v}}{dt} = \frac{d\vec{v}_r}{dt} - \vec{g}_{\left(\vec{r}\right)} - \vec{a}_T
\end{equation}
$\vec{v}_r$ is a function of time and space, so:
\begin{equation*}
    \frac{d\vec{v}_{r(t,\vec{r})}}{dt} = \frac{\partial\vec{v}_r}{\partial t} + \frac{\partial \vec{v}_r}{\partial\vec{r}}\underbrace{\frac{d\vec{r}}{dt}}_{\vec{v}} = \frac{\partial\vec{v}_r}{\partial t} + \frac{\partial \vec{v}_r}{\partial\vec{r}}\left(\vec{v}_r-\vec{v}_g\right)
\end{equation*}
The required velocity $\vec{v}_r$ also fulfill the equation of motion with no thrust:
\begin{equation*}
    \frac{d\vec{v}_r}{dt} = \frac{\partial\vec{v}_r}{\partial t} + \frac{\partial\vec{v}_r}{\partial \vec{r}}\vec{v}_r = \vec{g}_r
\end{equation*}
so:
\begin{equation}
    \frac{d\vec{v}_r}{dt} = \vec{g}_r - \frac{\partial\vec{v}_r}{\partial\vec{r}}\vec{v}_g
\end{equation}
The equation of motion becomes:
\begin{equation}
    \frac{d\vec{v}_g}{dt} =  \vec{g}_r - \frac{\partial\vec{v}_r}{\partial\vec{r}}\vec{v}_g -\vec{g}_{(\vec{r})} -\vec{a}_T = -\underbrace{\frac{\partial\vec{v}_r}{\partial\vec{r}}}_{C^*}\vec{v}_g-\vec{a}_T
\end{equation}
where $C^*$ is:
\begin{equation*}
    \begin{pmatrix}
        x_f\\y_f\\z_f
    \end{pmatrix} = \vec{0} = \Phi_{11(t_f,t)}\begin{pmatrix}
        x\\y\\z
    \end{pmatrix} + \Phi_{12(t_f,t)}\begin{pmatrix}
        \dot{x}_{req}\\\dot{y}_{req}\\\dot{z}_{req}
    \end{pmatrix}
\end{equation*}
\begin{equation*}
    \begin{pmatrix}
        \dot{x}_{req}\\\dot{y}_{req}\\\dot{z}_{req}
    \end{pmatrix} = -\Phi_{12(t_f,t)}^{-1}\Phi_{11(t_f,t)}\begin{pmatrix}
        x\\y\\z
    \end{pmatrix} = C^*_{(t_f,t)}\begin{pmatrix}
        x\\y\\z
    \end{pmatrix}
\end{equation*}
\begin{equation*}
    \Downarrow
\end{equation*}
\begin{equation*}
    C^*_{(t_f,t)} = -\Phi_{12(t_f,t)}^{-1}\Phi_{11(t_f,t)}
\end{equation*}

\vspace{1cm}
The Cross-Product fixed thrust control law:
\begin{equation}
    \vec{v}_g\parallel\frac{d\vec{v}_g}{dt}
\end{equation}
\begin{equation*}
    \Downarrow
\end{equation*}
\begin{equation*}
    \frac{d\vec{v}_g}{dt}\times\vec{v}_g=0
\end{equation*}
\begin{equation}
    \left(-C^*\vec{v}_g-\vec{a}_T\right)\times\vec{v}_g=0
\end{equation}
\begin{equation}
    \vec{a}_T\times\vec{v}_g=\underbrace{-\left(C^*\vec{v}_g\right)}_{\vec{p}}\times\vec{v}_g
\end{equation}
Multypling on the left by $\vec{v}_g$:
\begin{equation}
    \left(\vec{a}_T\times\vec{v}_g\right)\times\vec{v}_g = \left(\vec{p}\times\vec{v}_g\right)\times\vec{v_g}
\end{equation}
Expand according to:
\begin{equation*}
    \left(\vec{a}\times\vec{b}\right)\times\vec{c} = \left(\vec{a}\cdot \vec{c}\right)\vec{b} - \left(\vec{b}\cdot\vec{c}\right)\vec{a}
\end{equation*}
\begin{equation}
    \left(\vec{a}_T\cdot\vec{v}_g\right)\vec{v}_g - v_g^2\vec{a}_T = \left(\vec{p}\cdot\vec{v}_g\right)\vec{v}_g - v_g^2\vec{p}
\end{equation}
\begin{equation*}
    \Downarrow
\end{equation*}
\begin{equation}
    \vec{a}_T=\vec{p}+\frac{1}{v_g^2}\left(\left(\vec{a}_T\cdot\vec{v}_g\right)-\left(\vec{p}\cdot\vec{v}_g\right)\right)\vec{v}_g
\end{equation}
Defining $q=\vec{a}_T\cdot\hat{v}_g$ as the projection of the acceleration vector on the velocity to be gained, we get:
\begin{equation}
    \vec{a}_T=\vec{p}+\left(q-\left(\vec{p}\cdot\hat{v}_g\right)\right)\hat{v}_g
\end{equation}
The terminal braking will therefor be:
\begin{equation}
    \Delta v_f = \vec{0} - \begin{pmatrix}
        \dot{x}_{f}\\\dot{y}_{f}\\\dot{z}_{f}
    \end{pmatrix} = -\left(\Phi_{21(t_f,0)}\begin{pmatrix}
        x_0\\y_0\\z_0
    \end{pmatrix} + \Phi_{22(t_f,0)}\begin{pmatrix}
        \dot{x}_{req}\\\dot{y}_{req}\\\dot{z}_{req}
    \end{pmatrix}\right)
\end{equation}

\noindent To calculate the components of the thrust vector, we need to make sure that the solution exists and it is stable. according to Lyapunov:
\begin{equation}
    \begin{array}{ccc}
        \underbrace{U}_\text{Lyapunov function} = \frac{1}{2}\vec{v}_g^T\vec{v}_g \hspace{1cm} & U_{(0)}=0 & U_{\vec{v}_g}>0
    \end{array}
\end{equation}
For stabllity:
\begin{equation}
    \dot{U}=\vec{v}_g^T\left(-C^*\vec{v}_g-\vec{a}_T\right) = -\vec{v}_g^T\left(\left(\vec{a}_T\cdot\hat{v}_g\right)-\left(\vec{p}\cdot\hat{v}_g\right)\right)\hat{v}_g < 0
\end{equation}
we need the expression in the parentheses to be positive. Let's assume that $|\vec{a}_T|=a_{max}$ and after squaring both sides we get:
\begin{align}
    a_T^2&=p^2+2\left(q-\vec{p}\cdot\hat{v}_g\right)\left(\vec{p}\cdot\hat{v}_g\right)+\left(q-\vec{p}\cdot\hat{v}_g\right)^2 \\
    a_T^2&=p^2+q^2-\left(\vec{p}\cdot\hat{v}_g\right)^2 \\
    q&=\left(a_{max}^2-p^2+\left(\vec{p}\cdot\hat{v}_g\right)^2\right)^{\frac{1}{2}}
\end{align}

\noindent To sum up:
\begin{equation*}
    \colorbox{yellow}{$\vec{a}_T=\vec{p}+\left(q-\left(\vec{p}\cdot\hat{v}_g\right)\right)\hat{v}_g$}
\end{equation*}
where:
\begin{equation*}
    \colorbox{yellow}{$ \begin{array}{l}
        \vec{p} = -C^*\vec{v}_g \\
        q =\left(a_{max}^2-p^2+\left(\vec{p}\cdot\hat{v}_g\right)^2\right)^{\frac{1}{2}}
    \end{array} $}
\end{equation*}
By solving this system numerically, we obtain the approach trajectory.
\begin{figure}[H]
    \centering
    \includegraphics[width=1\textwidth]{images/graph5.png}
    \caption{3D figure of the orbit trajectory - case B}
    \label{fig:3D-plot-caseB}
\end{figure}
\begin{figure}[H]
    \centering
    \includegraphics[width=1\textwidth]{images/graph6.png}
    \caption{2D figure of the position over time - case B}
    \label{fig:2D-plot_over_time-caseB}
\end{figure}
\begin{figure}[H]
    \centering
    \includegraphics[width=0.9\textwidth]{images/graph7.png}
    \caption{Thrust acceleration components and total thrust acceleration over time - case B}
    \label{fig:accel_over_time-caseB}
\end{figure}
\begin{figure}[H]
    \centering
    \includegraphics[width=0.9\textwidth]{images/graph8.png}
    \caption{Total $\Delta v$ over time - case B}
    \label{fig:delta_v_over_time-caseB}
\end{figure}
We can see that we indeed acomplished the design criteria:
\begin{itemize}
    \item The thrust doesn't exceed the maximum available thrust.
    \item The miss distance at the final desired time is less than $1[\mathrm{m}]\ \left(0.997012\left[\mathrm{m}\right]\right)$.
    \item The terminal velocity is $1.19091\left[\displaystyle\frac{\mathrm{m}}{\mathrm{sec}}\right]$
\end{itemize}


\end{document}